\documentclass[]{article}
\usepackage{lmodern}
\usepackage{amssymb,amsmath}
\usepackage{ifxetex,ifluatex}
\usepackage{fixltx2e} % provides \textsubscript
\ifnum 0\ifxetex 1\fi\ifluatex 1\fi=0 % if pdftex
  \usepackage[T1]{fontenc}
  \usepackage[utf8]{inputenc}
\else % if luatex or xelatex
  \ifxetex
    \usepackage{mathspec}
  \else
    \usepackage{fontspec}
  \fi
  \defaultfontfeatures{Ligatures=TeX,Scale=MatchLowercase}
\fi
% use upquote if available, for straight quotes in verbatim environments
\IfFileExists{upquote.sty}{\usepackage{upquote}}{}
% use microtype if available
\IfFileExists{microtype.sty}{%
\usepackage{microtype}
\UseMicrotypeSet[protrusion]{basicmath} % disable protrusion for tt fonts
}{}
\usepackage[unicode=true]{hyperref}
\hypersetup{
            pdftitle={Projet RIM Linux},
            pdfborder={0 0 0},
            breaklinks=true}
\urlstyle{same}  % don't use monospace font for urls
\usepackage{color}
\usepackage{fancyvrb}
\newcommand{\VerbBar}{|}
\newcommand{\VERB}{\Verb[commandchars=\\\{\}]}
\DefineVerbatimEnvironment{Highlighting}{Verbatim}{commandchars=\\\{\}}
% Add ',fontsize=\small' for more characters per line
\newenvironment{Shaded}{}{}
\newcommand{\KeywordTok}[1]{\textcolor[rgb]{0.00,0.44,0.13}{\textbf{{#1}}}}
\newcommand{\DataTypeTok}[1]{\textcolor[rgb]{0.56,0.13,0.00}{{#1}}}
\newcommand{\DecValTok}[1]{\textcolor[rgb]{0.25,0.63,0.44}{{#1}}}
\newcommand{\BaseNTok}[1]{\textcolor[rgb]{0.25,0.63,0.44}{{#1}}}
\newcommand{\FloatTok}[1]{\textcolor[rgb]{0.25,0.63,0.44}{{#1}}}
\newcommand{\ConstantTok}[1]{\textcolor[rgb]{0.53,0.00,0.00}{{#1}}}
\newcommand{\CharTok}[1]{\textcolor[rgb]{0.25,0.44,0.63}{{#1}}}
\newcommand{\SpecialCharTok}[1]{\textcolor[rgb]{0.25,0.44,0.63}{{#1}}}
\newcommand{\StringTok}[1]{\textcolor[rgb]{0.25,0.44,0.63}{{#1}}}
\newcommand{\VerbatimStringTok}[1]{\textcolor[rgb]{0.25,0.44,0.63}{{#1}}}
\newcommand{\SpecialStringTok}[1]{\textcolor[rgb]{0.73,0.40,0.53}{{#1}}}
\newcommand{\ImportTok}[1]{{#1}}
\newcommand{\CommentTok}[1]{\textcolor[rgb]{0.38,0.63,0.69}{\textit{{#1}}}}
\newcommand{\DocumentationTok}[1]{\textcolor[rgb]{0.73,0.13,0.13}{\textit{{#1}}}}
\newcommand{\AnnotationTok}[1]{\textcolor[rgb]{0.38,0.63,0.69}{\textbf{\textit{{#1}}}}}
\newcommand{\CommentVarTok}[1]{\textcolor[rgb]{0.38,0.63,0.69}{\textbf{\textit{{#1}}}}}
\newcommand{\OtherTok}[1]{\textcolor[rgb]{0.00,0.44,0.13}{{#1}}}
\newcommand{\FunctionTok}[1]{\textcolor[rgb]{0.02,0.16,0.49}{{#1}}}
\newcommand{\VariableTok}[1]{\textcolor[rgb]{0.10,0.09,0.49}{{#1}}}
\newcommand{\ControlFlowTok}[1]{\textcolor[rgb]{0.00,0.44,0.13}{\textbf{{#1}}}}
\newcommand{\OperatorTok}[1]{\textcolor[rgb]{0.40,0.40,0.40}{{#1}}}
\newcommand{\BuiltInTok}[1]{{#1}}
\newcommand{\ExtensionTok}[1]{{#1}}
\newcommand{\PreprocessorTok}[1]{\textcolor[rgb]{0.74,0.48,0.00}{{#1}}}
\newcommand{\AttributeTok}[1]{\textcolor[rgb]{0.49,0.56,0.16}{{#1}}}
\newcommand{\RegionMarkerTok}[1]{{#1}}
\newcommand{\InformationTok}[1]{\textcolor[rgb]{0.38,0.63,0.69}{\textbf{\textit{{#1}}}}}
\newcommand{\WarningTok}[1]{\textcolor[rgb]{0.38,0.63,0.69}{\textbf{\textit{{#1}}}}}
\newcommand{\AlertTok}[1]{\textcolor[rgb]{1.00,0.00,0.00}{\textbf{{#1}}}}
\newcommand{\ErrorTok}[1]{\textcolor[rgb]{1.00,0.00,0.00}{\textbf{{#1}}}}
\newcommand{\NormalTok}[1]{{#1}}
\IfFileExists{parskip.sty}{%
\usepackage{parskip}
}{% else
\setlength{\parindent}{0pt}
\setlength{\parskip}{6pt plus 2pt minus 1pt}
}
\setlength{\emergencystretch}{3em}  % prevent overfull lines
\providecommand{\tightlist}{%
  \setlength{\itemsep}{0pt}\setlength{\parskip}{0pt}}
\setcounter{secnumdepth}{0}
% Redefines (sub)paragraphs to behave more like sections
\ifx\paragraph\undefined\else
\let\oldparagraph\paragraph
\renewcommand{\paragraph}[1]{\oldparagraph{#1}\mbox{}}
\fi
\ifx\subparagraph\undefined\else
\let\oldsubparagraph\subparagraph
\renewcommand{\subparagraph}[1]{\oldsubparagraph{#1}\mbox{}}
\fi

% set default figure placement to htbp
\makeatletter
\def\fps@figure{htbp}
\makeatother


\title{Projet RIM Linux}
\date{2017-01-05}

\begin{document}
\maketitle
\begin{abstract}
Projet Architecture et Système.
\end{abstract}

\section{Création du RIM-Linux}\label{cruxe9ation-du-rim-linux}

Younesse Kaddar

\emph{En ligne} :

\begin{itemize}
\item
  \url{http://younesse.net/Architecture-systeme/RIM-Linux}
\item
  Dépôt github : \url{https://github.com/youqad/RIM-Linux}
\item
  \href{/assets/ArchiSysteme/RIM-Linux.pdf}{version PDF}
\end{itemize}

\begin{center}\rule{0.5\linewidth}{\linethickness}\end{center}

\begin{enumerate}
\def\labelenumi{\arabic{enumi}.}
\item
  Le fichier de configuration du noyau utilisé est le fichier
  \texttt{config\_kernel}
\item
  L'image iso est \texttt{output.iso}
\end{enumerate}

\section{Le noyau Linux}\label{le-noyau-linux}

\begin{Shaded}
\begin{Highlighting}[]
\FunctionTok{mkdir} \NormalTok{-p build kernel RIM-Linux/rootbase}

\BuiltInTok{cd} \NormalTok{kernel}

\FunctionTok{wget} \NormalTok{https://cdn.kernel.org/pub/linux/kernel/v4.x/linux-4.9.tar.xz}
\FunctionTok{tar} \NormalTok{-xJf linux-4.9.tar.xz}
\FunctionTok{rm} \NormalTok{linux-4.9.tar.xz}
\BuiltInTok{cd} \NormalTok{linux-4.9}

\FunctionTok{make} \NormalTok{nconfig}
\end{Highlighting}
\end{Shaded}

\begin{itemize}
\item
  Décocher \texttt{Enable\ loadable\ module\ support}
\item
  Dans General Setup :

  \begin{itemize}
  \tightlist
  \item
    ne laisser que le support de l'initrd/l'initramfs
  \end{itemize}
\end{itemize}

\begin{Shaded}
\begin{Highlighting}[]
\FunctionTok{make} \NormalTok{-j11 bzImage}
\end{Highlighting}
\end{Shaded}

\section{BusyBox}\label{busybox}

\begin{Shaded}
\begin{Highlighting}[]
\BuiltInTok{cd} \NormalTok{../build}
\FunctionTok{wget} \NormalTok{http://www.busybox.net/downloads/busybox-1.26.0.tar.bz2}
\FunctionTok{bzip2} \NormalTok{-d busybox-1.26.0.tar.bz2}
\FunctionTok{tar} \NormalTok{xvf busybox-1.26.0.tar}
\FunctionTok{rm} \NormalTok{busybox-1.26.0.tar}
\BuiltInTok{cd} \NormalTok{busybox-1.26.0}
\FunctionTok{make} \NormalTok{menuconfig}
\FunctionTok{make} \NormalTok{-j11}
\FunctionTok{make} \NormalTok{-j11 install}

\FunctionTok{cp} \NormalTok{-a ./_install/* ../../RIM-Linux/rootbase}

\BuiltInTok{cd} \NormalTok{../../RIM-Linux/}


\FunctionTok{ldd} \NormalTok{rootbase/bin/busybox}
\end{Highlighting}
\end{Shaded}

qui renvoie :

\begin{verbatim}
    linux-vdso.so.1 =>  (0x00007ffc42a9b000)
    libm.so.6 => /lib/x86_64-linux-gnu/libm.so.6 (0x00007ffa52a51000)
    libc.so.6 => /lib/x86_64-linux-gnu/libc.so.6 (0x00007ffa5268c000)
    /lib64/ld-linux-x86-64.so.2 (0x00007ffa52d57000)
\end{verbatim}

Puis :

\begin{Shaded}
\begin{Highlighting}[]
\FunctionTok{mkdir} \NormalTok{-p rootbase/lib/x86_64-linux-gnu/ rootbase/lib64}

\FunctionTok{cp} \NormalTok{/lib/x86_64-linux-gnu/}\DataTypeTok{\{libm.so.6,libc.so.6\}} \NormalTok{rootbase/lib/x86_64-linux-gnu/}

\FunctionTok{cp} \NormalTok{/lib64/ld-linux-x86-64.so.2 rootbase/lib64/}


\FunctionTok{strip} \NormalTok{-v rootbase/lib/x86_64-linux-gnu/* rootbase/lib64/*}

\BuiltInTok{cd} \NormalTok{rootbase}

\FunctionTok{rm} \NormalTok{linuxrc}

\FunctionTok{ln} \NormalTok{-s bin/busybox init}
\end{Highlighting}
\end{Shaded}

On va récupérer la trame mise à notre disposition :

\begin{Shaded}
\begin{Highlighting}[]
\FunctionTok{cp} \NormalTok{~fhh/share/tp/projet/rim.linux.template.tbz2 ./}

\FunctionTok{bzip2} \NormalTok{-d rim.linux.template.tbz2}

\FunctionTok{tar} \NormalTok{xvf rim.linux.template.tar}

\FunctionTok{rm} \NormalTok{rim.linux.template.tar}
\end{Highlighting}
\end{Shaded}

Puis :

\begin{itemize}
\item
  On change le nom de l'utilisateur dans

  \begin{itemize}
  \tightlist
  \item
    \texttt{etc/passwd}
  \end{itemize}
\item
  Dans \texttt{etc/init.d/rcS} :

  \begin{itemize}
  \tightlist
  \item
    on décommente la ligne
    \texttt{busybox\ loadkmap\ \textless{}\ /etc/fr.kmap} pour activer
    le clavier français (on ajoutera un fichier \texttt{etc/fr.kmap} est
    bien présent)
  \item
    on change le nom d'hôte : \texttt{/bin/hostname\ RIM-Linux}
  \end{itemize}
\end{itemize}

\subsection{Archive cpio de l'initramfs et
isolinux}\label{archive-cpio-de-linitramfs-et-isolinux}

\begin{Shaded}
\begin{Highlighting}[]
\BuiltInTok{cd} \NormalTok{..}

\FunctionTok{mkdir} \NormalTok{-p rootcd/}\DataTypeTok{\{boot,isolinux\}}

\FunctionTok{cp} \NormalTok{../kernel/linux-4.9/arch/x86_64/boot/bzImage rootcd/boot/vmlinuz}

\FunctionTok{find} \NormalTok{rootbase/* -print }\KeywordTok{|} \FunctionTok{cpio} \NormalTok{-o -Hnewc }\OperatorTok{>} \NormalTok{root}

\FunctionTok{cat} \NormalTok{root }\KeywordTok{|} \FunctionTok{gzip} \NormalTok{-9 }\OperatorTok{>} \NormalTok{rootcd/boot/root.gz}

\FunctionTok{rm} \NormalTok{root}
\end{Highlighting}
\end{Shaded}

On récupère isolinux

\begin{Shaded}
\begin{Highlighting}[]

\BuiltInTok{cd} \NormalTok{../build}

\FunctionTok{wget} \NormalTok{https://www.kernel.org/pub/linux/utils/boot/syslinux/syslinux-6.03.tar.xz}

\FunctionTok{tar} \NormalTok{xJf syslinux-6.03.tar.xz}

\FunctionTok{rm} \NormalTok{syslinux-6.03.tar.xz}

\BuiltInTok{cd} \NormalTok{..}

\FunctionTok{cp} \NormalTok{build/syslinux-6.03/bios/core/isolinux.bin RIM-Linux/rootcd/isolinux}

\FunctionTok{cp} \NormalTok{build/syslinux-6.03/bios/com32/elflink/ldlinux/ldlinux.c32 RIM-Linux/rootcd/isolinux}
\end{Highlighting}
\end{Shaded}

Puis, on crée les fichiers

\begin{itemize}
\tightlist
\item
  \texttt{RIM-Linux/rootcd/isolinux/isolinux.cfg} :
\end{itemize}

\begin{verbatim}
display boot.txt

default 1

label 1
    kernel /boot/vmlinuz
    append initrd=/boot/root.gz
\end{verbatim}

\begin{itemize}
\tightlist
\item
  \texttt{RIM-Linux/rootcd/isolinux/boot.txt} : le message de bienvenue
\end{itemize}

\begin{Shaded}
\begin{Highlighting}[]
\BuiltInTok{cd} \NormalTok{RIM-Linux}

\ExtensionTok{mkisofs} \NormalTok{-o output.iso -b isolinux/isolinux.bin -c isolinux/boot.cat -no-emul-boot -boot-load-size 4 -boot-info-table rootcd}
\end{Highlighting}
\end{Shaded}

\section{Shell personnalisé}\label{shell-personnalisuxe9}

On se place à la racine du répertoire \texttt{RIM-Linux}.

Dans \texttt{rootbase/etc/inittab}, on remplace \texttt{/bin/sh} par
\texttt{/bin/shell}, en ayant pris soin de placer notre shell
personnalisé \texttt{shell} dans \texttt{rootbase/bin/shell}.

Il reste à ajouter les bibliothèques qu'utilise \texttt{shell} :

\begin{Shaded}
\begin{Highlighting}[]
\FunctionTok{ldd} \NormalTok{shell}
\end{Highlighting}
\end{Shaded}

renvoie

\begin{verbatim}
linux-vdso.so.1 =>  (0x00007ffe3ba9a000)
libreadline.so.6 => /lib/x86_64-linux-gnu/libreadline.so.6 (0x00007fb5dd018000)
libc.so.6 => /lib/x86_64-linux-gnu/libc.so.6 (0x00007fb5dcc53000)
libtinfo.so.5 => /lib/x86_64-linux-gnu/libtinfo.so.5 (0x00007fb5dca2a000)
/lib64/ld-linux-x86-64.so.2 (0x00007fb5dd25e000)
\end{verbatim}

\begin{Shaded}
\begin{Highlighting}[]
\FunctionTok{cp} \NormalTok{/lib/x86_64-linux-gnu/libreadline.so.6 rootbase/lib/x86_64-linux-gnu/}

\FunctionTok{cp} \NormalTok{/lib/x86_64-linux-gnu/libc.so.6 rootbase/lib/x86_64-linux-gnu/}

\FunctionTok{cp} \NormalTok{/lib/x86_64-linux-gnu/libtinfo.so.5 rootbase/lib/x86_64-linux-gnu/}
\end{Highlighting}
\end{Shaded}

(\texttt{cp\ /lib64/ld-linux-x86-64.so.2\ rootbase/lib64/} : déjà fait
précédemment).

Puis, ne pas oublier :

\begin{Shaded}
\begin{Highlighting}[]
\FunctionTok{strip} \NormalTok{-v lib/x86_64-linux-gnu/*}
\end{Highlighting}
\end{Shaded}

\section{Clavier}\label{clavier}

Dans \texttt{rootbase/etc/init.d/rcS}, on veillera bien à décommenter la
ligne \texttt{busybox\ loadkmap\ \textless{}\ /etc/fr.kmap}.

Puis :

\begin{itemize}
\tightlist
\item
  Si on a les droits administrateurs (ce qui n'est pas le cas sur les
  machines de l'ENS) :
\end{itemize}

\begin{Shaded}
\begin{Highlighting}[]
\FunctionTok{su}

\ExtensionTok{../build/busybox-1.26.0/_install/bin/busybox} \NormalTok{dumpkmap }\OperatorTok{>} \NormalTok{rootbase/etc/fr.kmap}
\end{Highlighting}
\end{Shaded}

\begin{itemize}
\tightlist
\item
  Sinon :
\end{itemize}

\begin{Shaded}
\begin{Highlighting}[]
\BuiltInTok{cd} \NormalTok{rootbase/etc/}

\FunctionTok{wget} \NormalTok{http://younesse.net/assets/ArchiSysteme/fr.kmap}

\FunctionTok{wget} \NormalTok{http://younesse.net/assets/ArchiSysteme/us.kmap}

\FunctionTok{wget} \NormalTok{http://younesse.net/assets/ArchiSysteme/uk.kmap}

\BuiltInTok{cd} \NormalTok{../..}
\end{Highlighting}
\end{Shaded}

Pour le message d'accueil indiquant comment passer le système en clavier
français ou anglais, créer :

\begin{itemize}
\item
  \texttt{rootbase/etc/motd} : le message de bienvenue

\begin{verbatim}
/  __\/ \/ \__/|  / \   / \/ \  /|/ \ /\\  \//
|  \/|| || |\/||  | |   | || |\ ||| | || \  /
|    /| || |  ||  | |_/\| || | \||| \_/| /  \
\_/\_\\_/\_/  \|  \____/\_/\_/  \|\____//__/\\
                    .--.
                   |o_o |
                   |:_/ |
                  //   \ \
                 (|     | )
                /'\_   _/`\
                \___)=(___/  
Use:    'loadkmap < /etc/us.kmap' (en_US),
        'loadkmap < /etc/uk.kmap' (en_UK)
    or  'loadkmap < /etc/fr.kmap' (fr_FR) to change the keyboard mapping

To display this message again : 'cat /etc/motd'
\end{verbatim}

  \begin{itemize}
  \item
    Puis :

    \begin{itemize}
    \item
      dans \texttt{rootbase/etc/init.d/rcS}, ajouter la ligne :

\begin{Shaded}
\begin{Highlighting}[]
\ExtensionTok{/etc/init.d/monscript}
\end{Highlighting}
\end{Shaded}
    \item
      créer \texttt{rootbase/etc/init.d/monscript} :

\begin{Shaded}
\begin{Highlighting}[]
\FunctionTok{cat} \NormalTok{/etc/motd}
\end{Highlighting}
\end{Shaded}
    \end{itemize}
  \end{itemize}
\end{itemize}

\section{Réseau}\label{ruxe9seau}

\begin{Shaded}
\begin{Highlighting}[]
\FunctionTok{mkdir} \NormalTok{-p rootbase/usr/share/udhcpc/}

\FunctionTok{cp} \NormalTok{../build/busybox-1.26.0/examples/udhcp/simple.script rootbase/usr/share/udhcpc/default.script}

\FunctionTok{chmod} \NormalTok{+x rootbase/usr/share/udhcpc/default.script}
\end{Highlighting}
\end{Shaded}

Puis ajouter les lignes suivantes au fichier
\texttt{rootbase/etc/init.d/rcS} :

\begin{Shaded}
\begin{Highlighting}[]
\ExtensionTok{ifconfig} \NormalTok{eth0 up}
\ExtensionTok{udhcpc} \NormalTok{-i eth0}
\end{Highlighting}
\end{Shaded}

Enfin :

\begin{Shaded}
\begin{Highlighting}[]
\BuiltInTok{cd} \NormalTok{rootbase}
\BuiltInTok{echo} \StringTok{"127.0.0.1      localhost"} \OperatorTok{>} \NormalTok{etc/hosts}
\BuiltInTok{echo} \StringTok{"localnet    127.0.0.1"} \OperatorTok{>} \NormalTok{etc/networks}
\BuiltInTok{echo} \StringTok{"RIM-Linux"} \OperatorTok{>} \NormalTok{etc/hostname}
\BuiltInTok{echo} \StringTok{"order hosts,bind"} \OperatorTok{>} \NormalTok{etc/host.conf}
\BuiltInTok{echo} \StringTok{"multi on"} \OperatorTok{>>} \NormalTok{etc/host.conf}
\end{Highlighting}
\end{Shaded}

\section{Configurations
supplémentaires}\label{configurations-suppluxe9mentaires}

Pour une configuration plus exhaustive, on peut créer les fichiers :

\begin{itemize}
\tightlist
\item
  \texttt{rootbase/etc/nsswitch.conf} :
\end{itemize}

\begin{Shaded}
\begin{Highlighting}[]
\CommentTok{# /etc/nsswitch.conf: GNU Name Service Switch config.}

\ExtensionTok{passwd}\NormalTok{:     files}
\ExtensionTok{group}\NormalTok{:      files}
\ExtensionTok{shadow}\NormalTok{:     files}

\ExtensionTok{hosts}\NormalTok{:      files dns}
\ExtensionTok{networks}\NormalTok{:   files}
\end{Highlighting}
\end{Shaded}

\begin{itemize}
\tightlist
\item
  \texttt{rootbase/etc/securetty} :
\end{itemize}

\begin{Shaded}
\begin{Highlighting}[]
\CommentTok{# /etc/securetty: List of terminals on which root is allowed to login.}

\ExtensionTok{console}

\CommentTok{# For people with serial port consoles}
\ExtensionTok{ttyS0}

\CommentTok{# Standard consoles}
\ExtensionTok{tty1}
\ExtensionTok{tty2}
\ExtensionTok{tty3}
\ExtensionTok{tty4}
\ExtensionTok{tty5}
\ExtensionTok{tty6}
\ExtensionTok{tty7}
\end{Highlighting}
\end{Shaded}

\begin{itemize}
\tightlist
\item
  \texttt{rootbase/etc/shells} :
\end{itemize}

\begin{Shaded}
\begin{Highlighting}[]
\CommentTok{# /etc/shells: valid login shells.}
\ExtensionTok{/bin/shell}
\ExtensionTok{/bin/sh}
\ExtensionTok{/bin/ash}
\ExtensionTok{/bin/hush}
\end{Highlighting}
\end{Shaded}

\begin{itemize}
\item
  \texttt{rootbase/etc/issue} :

\begin{verbatim}
RIM-Linux, Younesse Kaddar \r \l
\end{verbatim}
\item
  \texttt{rootbase/etc/busybox.conf} :
\end{itemize}

\begin{Shaded}
\begin{Highlighting}[]
\CommentTok{# /etc/busybox.conf: Busybox configuration.}

\NormalTok{[}\ExtensionTok{SUID}\NormalTok{]}
\CommentTok{# Allow command to be run by anyone.}
\FunctionTok{su} \NormalTok{= ssx root.root}
\FunctionTok{passwd} \NormalTok{= ssx root.root}
\ExtensionTok{loadkmap} \NormalTok{= ssx root.root}
\FunctionTok{mount} \NormalTok{= ssx root.root}
\ExtensionTok{reboot} \NormalTok{= ssx root.root}
\ExtensionTok{halt} \NormalTok{= ssx root.root}
\end{Highlighting}
\end{Shaded}

⟶ on veillera aussi à protéger ce fichier :

\begin{Shaded}
\begin{Highlighting}[]
\FunctionTok{chmod} \NormalTok{600 rootbase/etc/busybox.conf}
\end{Highlighting}
\end{Shaded}

\section{Utilitaires}\label{utilitaires}

Les scripts \texttt{cpio\_creator}, \texttt{kernel\_copy},
\texttt{rebuild\_iso}, \texttt{send\_home} créés à la racine de
\texttt{RIM-Linux/} automatisent certaines tâches répétitives effectuées
pendant les tests.

\begin{itemize}
\tightlist
\item
  \texttt{cpio\_creator} :
\end{itemize}

\begin{Shaded}
\begin{Highlighting}[]
\CommentTok{#!/bin/sh}
\VariableTok{HERE=$(}\BuiltInTok{pwd}\VariableTok{)}
\VariableTok{ROOTBASE=}\NormalTok{./rootbase}
\BuiltInTok{echo} \StringTok{"The path is : }\VariableTok{$ROOTBASE}\StringTok{."}
\BuiltInTok{echo} \StringTok{"Creating ..."}
\BuiltInTok{cd} \VariableTok{$ROOTBASE}
\FunctionTok{find} \NormalTok{./* -print }\KeywordTok{|} \FunctionTok{cpio} \NormalTok{-o -Hnewc }\OperatorTok{>} \VariableTok{$HERE}\NormalTok{/root}
\BuiltInTok{cd} \VariableTok{$HERE}
\FunctionTok{cat} \NormalTok{root }\KeywordTok{|} \FunctionTok{gzip} \NormalTok{-9 }\OperatorTok{>} \NormalTok{root.gz}
\FunctionTok{rm} \NormalTok{root}
\FunctionTok{mv} \NormalTok{root.gz rootcd/boot}
\BuiltInTok{echo} \StringTok{"Done."}
\end{Highlighting}
\end{Shaded}

\begin{itemize}
\tightlist
\item
  \texttt{kernel\_copy} :
\end{itemize}

\begin{Shaded}
\begin{Highlighting}[]

\CommentTok{#!/bin/sh}
\FunctionTok{cp} \NormalTok{../kernel/linux-4.9/arch/x86_64/boot/bzImage rootcd/boot/vmlinuz}
\end{Highlighting}
\end{Shaded}

\begin{itemize}
\item
  \texttt{rebuild\_iso\ {[}options{]}} : crée l'image iso du RIM Linux
  et la lit avec \texttt{qemu}

  \begin{itemize}
  \tightlist
  \item
    options :

    \begin{itemize}
    \tightlist
    \item
      \texttt{-kernel} : copie le noyau \texttt{bzImage} créé dans le
      répertoire \texttt{../kernel}
    \item
      \texttt{-kvm} : lance \texttt{qemu} avec l'option
      \texttt{-enable-kvm}
    \end{itemize}
  \end{itemize}
\end{itemize}

\begin{Shaded}
\begin{Highlighting}[]
\CommentTok{#!/bin/sh}
\ExtensionTok{./cpio_creator}
\KeywordTok{if}\BuiltInTok{ [} \StringTok{"}\VariableTok{$1}\StringTok{"} \OtherTok{==} \StringTok{"-kernel"}\BuiltInTok{ ]} \KeywordTok{||}\BuiltInTok{ [} \StringTok{"}\VariableTok{$2}\StringTok{"} \OtherTok{==} \StringTok{"-kernel"}\BuiltInTok{ ]} \NormalTok{; }\KeywordTok{then}
  \ExtensionTok{./kernel_copy}
  \BuiltInTok{echo} \StringTok{"Kernel copied"}
\KeywordTok{else}
  \BuiltInTok{echo} \StringTok{"Kernel not copied"}
\KeywordTok{fi}
\ExtensionTok{mkisofs} \NormalTok{-o output.iso -b isolinux/isolinux.bin -c isolinux/boot.cat -no-emul-boot -boot-load-size 4 -boot-info-table rootcd}

\KeywordTok{if}\BuiltInTok{ [} \StringTok{"}\VariableTok{$1}\StringTok{"} \OtherTok{==} \StringTok{"-kvm"}\BuiltInTok{ ]} \KeywordTok{||}\BuiltInTok{ [} \StringTok{"}\VariableTok{$2}\StringTok{"} \OtherTok{==} \StringTok{"-kvm"}\BuiltInTok{ ]} \NormalTok{; }\KeywordTok{then}
  \BuiltInTok{echo} \StringTok{"Qemu : KVM enabled"}
  \ExtensionTok{qemu-system-x86_64} \NormalTok{-m 4G -enable-kvm -cdrom output.iso}
\KeywordTok{else}
  \BuiltInTok{echo} \StringTok{"Qemu : KVM disabled"}
  \ExtensionTok{qemu-system-x86_64} \NormalTok{-m 4G -cdrom output.iso}
\KeywordTok{fi}
\end{Highlighting}
\end{Shaded}

\begin{itemize}
\tightlist
\item
  \texttt{send\_home} :
\end{itemize}

\begin{Shaded}
\begin{Highlighting}[]
\CommentTok{#!/bin/sh}
\FunctionTok{rm} \NormalTok{-rf ~/RIM-Linux}
\FunctionTok{cp} \NormalTok{-R ../RIM-Linux/ ~/}
\end{Highlighting}
\end{Shaded}

\end{document}
